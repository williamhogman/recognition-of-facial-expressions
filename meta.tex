% Filling in this bibliographic information facilitates the
% processing of this document.

% Insert linebreaks if necessary

% Just leave the posistion blank for any information that doesn't
% apply for your thesis, e.g. if your thesis doesn't have a subtitle:
%
% \newcommand{\thesisSubtitle}{}




% Abstract and titelpage

\firstAuthorFirstName{William}                % First author given name
\firstAuthorSurname{Rudenmalm}                 % First author surname


\firstAuthorEmail{wru@dsv.su.se}              % First author's e-mail address

\thesisTitle{A Multitask, Software-based Emotional Intelligence Test}                   % The title of the thesis
\thesisSubtitle{} % The subtitle of the thesis

\thesisSubject{Human-Computer Interaction}   % May be changed
                                                % to e.g. Human-Computer
                                                % Interaction or other
                                                % suitable for your thesis
\thesisIsKind{Master}                           % Change to Bachelor if suitable
\theYear{2016}
\thesisCred{15}                                 % Change to 15 if Bachelor
\thesisAdvisor{Mats Wiklund}
\thesisAssistantAdvisor{Peter Mozelius}                       % Name of Assistant Advisor,
                                                % if you have one
\thesisExternalAdvisor{}                        % Name of External Advisor,
                                                % if you have one

\thesisReviewer{Panagiotis Papatreou}
\semester{Spring}
\swedishTitle{Ett emotionellt intelligenstest med flera uppgifter}

                                               % The abstract text comes
                                               % here. Not more than 300
                                               % words. No empty lines.

\abstracttext{Emotional intelligence, the ability to socially adapt to the affects of oneself and others is a fundamental social skill. Researchers have defined the term in different ways and it is at present somewhat controversial. Likewise, researchers have created many emotional intelligence tests, all tightly coupled to their respective definitions. This thesis-work created a new software-based test, combining two earlier approaches, SPEC and METT.evaluated the software using an experiment where 81 university students took the test once and 31 took it twice. Analyzing the results, a high degree of reliability for SPEC, low correlations between SPEC and METT, and a low test-retest reliability for SPEC. Based on this, this thesis concludes: METT is highly reliable; SPEC measures a different quality than does METT, but is an unreliable instrument. Combining tests proved successful, however performance of SPEC was poor. A possible aim for future research is finding a more reliable alternative to SPEC.}

\keywords{emotional intelligence, emotion, psychometrics, software}
