% Created 2015-04-18 Sat 16:27
\documentclass[11pt]{article}
\usepackage[utf8]{inputenc}
\usepackage[T1]{fontenc}
\usepackage{fixltx2e}
\usepackage{graphicx}
\usepackage{longtable}
\usepackage{float}
\usepackage{wrapfig}
\usepackage{rotating}
\usepackage[normalem]{ulem}
\usepackage{amsmath}
\usepackage{textcomp}
\usepackage{marvosym}
\usepackage{wasysym}
\usepackage{amssymb}
\usepackage{hyperref}
\tolerance=1000
\date{\today}
\title{result\_implementation}
\hypersetup{
  pdfkeywords={},
  pdfsubject={},
  pdfcreator={Emacs 24.4.50.1 (Org mode 8.2.6)}}
\begin{document}

\maketitle
\tableofcontents

The resulting artifacts in comprised of a number of parts. These
are, the METT module, the questionnaire module, run-time
configuration, deployment scripts and text assets. These components
are replaceable with other components on an as-needed basis.

\begin{figure}[htpb]
\centering
\begin{tikzpicture}[
  scale=2.0,
  align=center,node distance=2cm
  every node/.style={draw,circle,very thick}
]

\graph [distance=0.5in, sibling distance=0.5in] {
  Experiment -> {
    METT,
    Questionnarie,
    Config,
    Questionnarie -> UI,
    Config -> Util,
    UI -> Util,
  }
};
\end{tikzpicture}
\end{figure}

\section{Software support for artefact configuration}
\label{sec-1}
The Pyton modules expects configuration in the JavaScript Object
Notation (JSON). These configuration artifacts consists of three
separate files.

One with overall configuration global applicable to each trial and
question of the experiment. This file is stored with the filename
\texttt{options.json}.

\begin{lstlisting}[caption=options.json Configuration file]
{
  "NameOfTheExperiment": {
    "forward_mask_frames": 40
    "backward_mask_frames": 40,
    "expression_frames": 12,
    "intertrial_interval": 2,
 }
}
\end{lstlisting}

The second file, \texttt{trials.json} specifies paths to image files for masks
and stimuli as well as correct responses for each individual trial
of the METT paradigm.

\begin{lstlisting}[caption=\textt{trials.json} trials file]
{
  "NameOfTheExperiment": [
    {"stim": "PathToStimulus.JPG", "neu": "PathToMask.png", "correct": 5},
    // ...
  ]
}
\end{lstlisting}

Finally, a third JSON file specifies the questions for
SPEC.

\begin{lstlisting}[caption=\textt{trials.json} trials file]
{
  "NameOfTheExperiment": [
     {
       "question_id": 25,
       "responses": {
         "1": "Option One",
         // ...
       },
       "text": "Question"
     }
   // ..
  ]
}
\end{lstlisting}

\section{Tools for artefact deployment}
\label{sec-2}
To simplify the configuration and use of the system a number of
utility scripts were developed, including scripts for generating
subsets of the RaFD, starting the system in a given configuration
state, and running a demonstration mode of the software. These
scripts are not run in the normal operations of the appliation,
however, they are useful in setting up an environment that can then
be reused.

\section{Text assets and translation}
\label{sec-3}
For the purposes of making the application locablizable to different
languages, a system for text assets was created. Text assets for two
languages, Swedish and English were created. These assets are used
to progrmatically translate the application.

\section{Initialization module}
\label{sec-4}
The initialization module of the system manages initializing the
system on load. The first task of the is to collect the subject
identifier and experimenter initials from the experimenter. Once
this is performed, the current time and hostname are also collected
to help with diagnostics. The module then delegates to the
configuration modules which then proceeds to deserialize
configuration files and load the configuration files into
memory. After loading files passes-on the trials to the METT module
which displays performs the trials. When the METT module has
performed as its trials, the initialization module invokes the
question module, passing on the questions. When both METT and
questions have completed, collected responses are written to CSV.

\section{METT module}
\label{sec-5}
Before the experiment is started, the module shuffles the order of
trials used in the experiment. This results in each participant
experiencing the trials in a psuedo-random, non-systematic order
of trials.

The trials are then invoked in order, upon invocation the trial
handler loads the relevant images from disk, showing a blank screen
while loading. Once the images have been loaded the picture of the
neutral expression is showed for a fixed, but configurable number of
frames. Once the configured number of frames have rendered, the
emotional picture is displayed, for a configurable but different
number of frames. Then, finally the neutral image is shown for the
another configured number of frames.

Once the pictures have been displayed, the system displays an input
prompt and starts accepting keyboard input from the user. Once an
answer has been collected , the screen is then cleared. The flow of
execution then halts for a configurable number of seconds before the
trial yields control back to experiment.

Once the set of trials has been completed, the system, takes the
trial data and combines it with the response data. This results in a
one-row per trial data-set, which is then serialize and saved to
disk as a comma separated values file.

\section{Questionarie module}
\label{sec-6}
After the METT module has completed its work the questionarie module
is initialized, each question consists of three components, a
\texttt{question-id} which is simply logged to the output, a text description
for prompting the user, and finally a mapping, \texttt{keymap}, between keyboard key
names and a text-string containing the question text.

Each question present in configuration is then presented in
order. The presentation of a question is performed as follows,
first, the question text and the response options are rendered to
the screen as text. Keyboard input is read. If the input read
matches a response alternative, the key and response time are then
stored in memory. This process is repeated untill each configured
question has been performed.  has been processed. After the
questinarie task is completed the results are saved to disk in the
comma separated value format.
% Emacs 24.4.50.1 (Org mode 8.2.6)
\end{document}
